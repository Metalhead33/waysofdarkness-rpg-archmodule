\documentclass[openany,11pt,a4paper]{book}
\usepackage{graphicx}
\usepackage[utf8x]{inputenc}
\usepackage{multirow}
\usepackage{blindtext}
\usepackage{graphicx}
\usepackage{numprint}
\usepackage{listings}
\usepackage{xcolor}
\usepackage{amsmath}
\usepackage{mathtools}
\usepackage{float}
\usepackage{caption}
\usepackage{hyperref}
\usepackage{enumitem}
\usepackage{newtxtext,newtxmath}
\usepackage[square,sort,comma,numbers]{natbib}
\usepackage[a4paper, margin=1.5cm]{geometry}
\graphicspath{ {./images/} }
\author{Metalhead33}
\title{Ways of Darkness\\
   \normalsize A module for the World of Artograch RPG system}

\definecolor{mGreen}{rgb}{0,0.6,0}
\definecolor{mGray}{rgb}{0.5,0.5,0.5}
\definecolor{mPurple}{rgb}{0.3,0,0.3}
\definecolor{backgroundColour}{rgb}{0.95,0.95,0.92}

\begin{document}
\newcommand{\MammalRace}[1]{Being bipedal mammals without tails, {#1} have two hands and two legs, meaning that they can wield only two one-handed weapons \textit{(or a one-handed weapon and a shield)} or a single two-handed weapon at the same time. Just like with other bipedal mammalian races, reaching zero hitpoints on the arms or legs renders those limbs unusable \textit{(and disables the corresponding item slots)}, while reaching zero hitpoints at the torso or head means death - or unconsciousness, if house rules rule out death.}
\newcommand{\Bonus}[1]{\textcolor{green}{\textbf{+{#1} bonus}}}
\newcommand{\Malus}[1]{\textcolor{red}{\textbf{-{#1} malus}}}
\newcommand{\DimorphismBB}[5]{Male {#1} get a \Bonus{{#2}} to {#3}, while Female {#1} get a \Bonus{{#4}} to {#5}.}
\maketitle
\tableofcontents
\chapter*{Preface}
You are currently reading the documentation of the \textbf{Ways of Darkness module for} the \textbf{World of Artograch RPG system}. The author of this document presents you all the information within this document with the assumption that you have read the \textbf{World of Artograch Ruleset}, and thus are familiar with the rules it presented. Contents of the aforementioned document are expected to be referenced in this document.
\addcontentsline{toc}{chapter}{Preface}
\chapter{Introduction to the Occident}
\includegraphics[width=\textwidth,height=\textheight,keepaspectratio]{Occident_Borders}
\newpage
The \textbf{Occident} is the part of \textbf{Artograch} with perhaps the most dynamic history. While even the Occident shows a similar tendency as the Orient to have large countries dominated by a single race, it is slightly more nuanced in the Occident, which has had a history of territories changing hands. While the Orient has always been characterized by the rule of centralized and homogenous kingdoms, and highly bureaucratic and usually equally homogenous, often quasi-despotic empires, the Occident has always been a place where central authorities held only limited amount of power, with provinces enjoying high autonomy.\newline
The various races of the Occident are divided into two major groups: the Elven races \textit{(Humans, High Elves, Wood Elves, Dark Elves, Orcs)} being descended from Oriental invaders; and the indigenous races \textit{(Goblins, Ogres, Lizardmen, Halflings, Gnomes and Dwarves)} whose population was decimated by the arrival of the earlier. Out of the two, the earlier are clearly the dominant force on the continent, with three out of the four dominant powers in the Occident being dominated by an Elven race - Etrand by Humans, Froturn by High Elves, Dragoc by Wood Elves.\newline
Currently, as of 831 AEKE \textit{(831 years after the establishment of the Kingdom of Etrand)}, the Occident is at a tipping point: the three aforementioned great powers are at a three-way cold war with each other, with tensions being at an all-time high, while the previously unmentioned fourth, Gabyr, is sitting in the shadows watching it all with popcorn in their hands, while slowly expanding their trade empire in the Orient. Unseen for roughly five and a half centuries, Demons have been sighted lurking around, no doubt planting their own secret agents among the authorities of the aforementioned states. It is said that even the Orcs of Brutang are getting quite unruly, while the Empire of Neressa continues its long-practiced isolationist policies and the Principality of Artaburro tries to wiggle between the Kingdoms of Froturn, Dragoc and Etrand.
\chapter{Playable races}
\section{Humans}
\includegraphics{Average_Humans}\newline
\textbf{Humans} are the dominant race of the \textbf{Kingdom of Etrand} and its vassal, the \textbf{Earldom of Etrancoast}, albeit they are also present in high numbers in the \textbf{Principality of Gabyr}, constituting slightly more than half of its population, albeit they're not politically dominant there. In spite of their short lifespan - less than one century! - and their clear lack of pointy ears, humans are in fact an Elven race, descendants of those Ancestral Elves who crossed over from the Orient to colonize the Occident, making them a sister-race to the Wood Elves and High Elves, who also directly evolved out of those Ancestral Elves.\newline
\begin{tabular}{|c|c|c|}
\hline
 & \textbf{Min} & \textbf{Max} \\ \hline
\textbf{Strength} & 8 & 18 \\ \hline
\textbf{Endurance} & 8 & 18 \\ \hline
\textbf{Dexterity} & 8 & 18 \\ \hline
\textbf{Intelligence} & 8 & 18 \\ \hline
\textbf{Willpower} & 8 & 18 \\ \hline
\textbf{Charisma} & 8 & 18 \\ \hline
\end{tabular}\newline
\DimorphismBB{Humans}{1}{Strength and Endurance}{1}{Dexterity and Charisma} \MammalRace{humans}
\section{High Elves}
\includegraphics{High_Elves}\newline
\textbf{High Elves} are the dominant race of the \textbf{Kingdom of Froturn} and have a plurality in the \textbf{Empire of Neressa}, albeit their political influence expands far beyond, to the point that the Humans have adopted their religion over eight centuries ago! They have rather long lifespans, theoretically ten times that of a human \textit{(and they're also typically taller than humans, not to mention their sharp ears)}, ageing at one tenth a human's rate after reaching the age of eighteen. They descendants of those Ancestral Elves who crossed over from the Orient to colonize the Occident, making them a sister-race to the Wood Elves and Humans, who also directly evolved out of those Ancestral Elves.\newline
\begin{tabular}{|c|c|c|}
\hline
 & \textbf{Min} & \textbf{Max} \\ \hline
\textbf{Strength} & 8 & 18 \\ \hline
\textbf{Endurance} & 7 & 17 \\ \hline
\textbf{Dexterity} & 9 & 19 \\ \hline
\textbf{Intelligence} & 8 & 18 \\ \hline
\textbf{Willpower} & 8 & 18 \\ \hline
\textbf{Charisma} & 9 & 19 \\ \hline
\end{tabular}\newline
\DimorphismBB{High Elves}{1}{Strength}{1}{Charisma} \MammalRace{High Elves}
\section{Wood Elves}
\includegraphics{Wood_Elves}\newline
\textbf{Wood Elves} are the dominant race of the \textbf{Kingdom of Dragoc}. They have rather long lifespans, theoretically ten times that of a human \textit{(and they're also typically taller than humans, not to mention their sharp ears)}, ageing at one tenth a human's rate after reaching the age of eighteen. They descendants of those Ancestral Elves who crossed over from the Orient to colonize the Occident, making them a sister-race to the High Elves and Humans, who also directly evolved out of those Ancestral Elves.\newline
\begin{tabular}{|c|c|c|}
\hline
 & \textbf{Min} & \textbf{Max} \\ \hline
\textbf{Strength} & 8 & 18 \\ \hline
\textbf{Endurance} & 7 & 17 \\ \hline
\textbf{Dexterity} & 9 & 19 \\ \hline
\textbf{Intelligence} & 8 & 18 \\ \hline
\textbf{Willpower} & 8 & 18 \\ \hline
\textbf{Charisma} & 8 & 18 \\ \hline
\end{tabular}\newline
\DimorphismBB{Wood Elves}{2}{Strength}{2}{Charisma} \MammalRace{Wood Elves}
\section{Dark Elves}
\includegraphics{Dark_Elves}\newline
\textbf{Dark Elves} are a race without a unified country, living in underground clans in dungeon-cities below the the ground, though also being present in several surface states - namely the Kingdoms of Etrand and Froturn - in high numbers as diaspora. They have rather long lifespans, theoretically ten times that of a human \textit{(and they're also typically taller than humans, not to mention their sharp ears and blue-ish grey skin)}, ageing at one tenth a human's rate after reaching the age of eighteen. They descendants of Wood Elves and High Elves who were banished after experimenting with the dark arts, creating their own exodus, where the environment - and the new religion - has altered their appearence.\newline
\begin{tabular}{|c|c|c|}
\hline
 & \textbf{Min} & \textbf{Max} \\ \hline
\textbf{Strength} & 8 & 18 \\ \hline
\textbf{Endurance} & 7 & 17 \\ \hline
\textbf{Dexterity} & 10 & 20 \\ \hline
\textbf{Intelligence} & 8 & 18 \\ \hline
\textbf{Willpower} & 8 & 18 \\ \hline
\textbf{Charisma} & 8 & 18 \\ \hline
\end{tabular}\newline
\DimorphismBB{Dark Elves}{1}{Strength}{1}{Charisma} \MammalRace{Dark Elves}
\section{Half-Elves}
\includegraphics{Half_Elves}\newline
\textbf{Half-Elves} are hybrids of Humans and any of the aforementioned Elven races \textit{(High Elves, Wood Elves, Dark Elves)}. To be specific, this race only includes only \textit{``clean hybrids''} - as in, people with 45-55\% Human blood and 45-55\% Elven blood. People where one side dominates - whether it's the Human side or Elven side - are classified as \textit{``unclean hybrids''}, and thus get shoehorned into the Human or Elven race.\newline
\begin{tabular}{|c|c|c|}
\hline
 & \textbf{Min} & \textbf{Max} \\ \hline
\textbf{Strength} & 8 & 18 \\ \hline
\textbf{Endurance} & 8 & 18 \\ \hline
\textbf{Dexterity} & 9 & 19 \\ \hline
\textbf{Intelligence} & 8 & 18 \\ \hline
\textbf{Willpower} & 8 & 18 \\ \hline
\textbf{Charisma} & 8 & 18 \\ \hline
\end{tabular}\newline
\DimorphismBB{Half-Elves}{1}{Strength}{1}{Charisma} \MammalRace{Half-Elves}
\section{Half-Orcs}
\includegraphics{Half_Orcs}
\section{Orcs}
\includegraphics{Orks}
\section{Ogres}
\includegraphics{Ogres}
\section{Goblins}
\includegraphics{Goblins}
\section{Halflings}
\includegraphics{Hobbits}
\section{Gnomes}
\includegraphics{Gnomes}
\section{Dwarves}
\includegraphics{Dwarves}
\section{Lizardmen}
\includegraphics{Average_Lizardman}
\chapter{Classes}
\section{Fighter Type Classes}
\subsection{The \textit{(Occidental)} Mercenary}
\subsection{The Headhunter}
\subsection{The Footman}
\subsection{The Sharpshooter}
\section{Rogue Type Classes}
\subsection{The Thief}
\subsection{The Assassin}
\subsection{The Smuggler}
\subsection{The Scout}
\section{Magician Type Classes}
\subsection{The Mage}
\subsection{The Battlemage}
\subsection{The Spellthief}
\subsection{The Warlock}
\section{Cleric Type Classes}
\subsection{The Priest}
\subsection{The Inquisitor}
\subsection{The Knight}
\subsection{The Monk}
\section{Druid Type Classes}
\subsection{The Shaman}
\subsection{The Ranger}
\subsection{The Bard}
\subsection{The Herbalist}
\chapter{Feats}
\chapter{Weapons n' Armour}
\chapter{Magicks}
\end{document}
